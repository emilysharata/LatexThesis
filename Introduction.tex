\label{sec:introduction}
\chapter{Introduction}
The focus of this research is to utilize artificial neural networks to classify a collection of job offers advertised in Switzerland as environmentally related or not, as well as to use Natural Language Processing (NLP) techniques to extract the work 
activities and skills from these said job offers. In particular, Google's pre-trained Universal Sentence Encoder (USE) is utilized to measure textual semantic similarity between the job offers and pre-defined databases of environmentally related skills and activities.

IE is a form of NLP in which specific types of information can be identified from text. 
Information such as user-defined keywords, relationships, dates, and events can be identified, classified, 
and extracted from documents, and then organized in a spreadsheet or database. IE is not to be confused with 
Information Retrieval (IR), in which a user generates a specific query and selects a relevant subset of documents 
from a larger collection~\cite{gaizauskas-wilks-1998-information}. Therefore, the focus of IE is on the gathering of specific information 
from texts or documents, while IR gathers the relevant documents from a larger corpus. In practice, these
two techniques are often used together in language processing. 

The history of modern IE can be traced back to the late 1980’s. The field developed quickly when the United States
Defense Advanced Research Projects Agency funded a large project aimed at extracting important information from naval 
messages in the US Navy~\cite{grishman_2019}. Experimental IE systems would be evaluated on their ability to accurately 
extract the messages of interest. During this time, the US government would host annual Message Understanding 
Conferences (MUCs) where various academic and industrial labs would compete to build the  best performing IE systems. 
Following the research surrounding the naval messages, these teams frequently worked with domains such as newswire articles. 
These historical conferences are of note because they represent the first widespread attempts at the development of large-scale, 
organized NLP systems. A result of these early years of research was the development of one of the first commercial IE 
systems called ATRANS, which could automatically process money transfer messages between banks~\cite{gaizauskas-wilks-1998-information}. 
A news fact extractor system called JASPER was developed for Reuters to help journalists quickly gather key facts to assist 
them in writing headlines more quickly. Larger, more complex IE systems developed in the 1990’s followed. These included systems 
such as vehicle fault report summarization, academic journal databases, employment opportunity databases, and police report 
summarization. While IE was indeed expanding into more diverse domains, the accuracy and recall of these systems were 
typically only around 50\%~\cite{gaizauskas-wilks-1998-information}.


In the 30 years since the inception of IE, the landscape of NLP has changed dramatically and the number of potential 
applications of IE has increased exponentially. As the Information Age has shifted into the Machine Learning Age, where 
crowdsourced or User Generated Data (UGD) produced by billions of users of smartphone, internet, and internet of things 
devices have rapidly improved Artificial Intelligence (AI) technology, we have begun to enter a time where computers can 
overtake human intelligence. AI technology surrounding natural text and language has advanced hugely in the past 10 years 
through supervised and unsupervised machine learning. It has given rise to disciplines and technologies such as sentiment 
analysis, where machines are able to recognize subtle nuances in text such as sarcasm and irony, virtual assistants, machine 
translation, document summarization, auto-correct, speech recognition, and text extraction. Everyday voice assistants including 
Amazon’s Alexa and Apple’s Siri are built on recent advances in NLP.  Within business settings, these tools can be used to help 
monitor and manage large amounts of unstructured text data such as email, social media conversations, customer service channels, 
market research queries, survey responses, and more. By automating many NLP tasks, companies can quickly and cost effectively 
gain better control over their data and drive better business decisions. 

