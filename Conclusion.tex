\chapter{Conclusions and future prospects}
\label{sec:conclusions}
This thesis has presented a method for classifying job offers advertised in Switzerland and a method for identifying the relevant skills and activities in the job offers. A general software framework has been developed for this task, found in Ref.~\cite{emilysharata}. This framework was designed to enable each step in the analysis tasks of this research project. In particular, the job offers were cleaned of non-alphanumeric characters and converted to numeric data useful for automated natural language processing tasks before applying machine-learning techniques and interpreting the results.

Two neural network models were built for the classification task. The more complex model was tuned by changing the learning rate optimizer method and the number of epochs. The neural network using the RMS Prop optimizer and three epochs performed the best with an F-measure of 0.803. Google's Universal Sentence Encoder was used for the skills and activities identification. By measuring the cosine distance between the encoded sentences of the job offer texts and the sentences of the skills and activities references, the skills and activities with the highest semantic similarity to job offer, and thus the most relevant, were identified. By inspecting the results for a small, random subset of the jobs and tasks, it has been demonstrated that this technique is viable for the categorization and association of jobs to skills and activities.

There were several challenges posed by this research question. The job offers data set had many complications. This data set compiled by X28 was intended to only contain environmentally related job offers, but also contained job offers from unrelated sectors as well as incomplete or unintelligible job descriptions. This created a need to develop a classifier to identify whether a job offer was environmentally related or not. For the classification task, there were only 536 labeled job offers. Labeling the data was labor and time intensive and therefore the number of labeled data was extremely limited. This made the test and train data sets relatively small, making it difficult to adequately train and deploy the model.

For the semantic textual similarity measures between the job offers text and the activities and skills texts, the main challenge was the lack training data. Without this, it was not possible to quantitatively score the performance of the USE model's ability to identify the relative skills and activities in the text. Given the large number of reference skills and activities, as well as the length of the text of some of the job offers, the labeling of the data for semantic similarity would be extremely challenging and time consuming. Therefore, only a qualitative manual inspection of the results was possible. 

Despite the limitations of the data set, this research has demonstrated incredible abilities of transfer learning with pre-trained models. By manually inspecting a small subset of the associated activities and skills, it is clear that the USE does encode relevant information, and that this technique is a viable one to simplify and automate the association of environmental jobs to tasks and skills. In addition, this work has established a usable software setup to continue to explore the data set, and a baseline procedure to accomplish this task.

Moving forward, this work could be expanded in several ways. A larger labeled data set would facilitate the development of a more robust neural net classification model, and would allow for the possibility to score the USE models quantitatively with the F-measure. As is the challenge with many NLP tasks, it is difficult to quantitatively evaluate their performances. Another way future research could be expanded is through a more thorough cleaning of the job offers. Despite removing special characters and translating the text, many job offers still contained irrelevant numbers, URLs, or other unintelligible text. Furthermore, it could be worthwhile to develop a method to remove the part of the job offer that describes the company, so that the focus of the semantic similarity measure is only between the part of the text that describes the job function and the skills and activities. This could in principal also improve the accuracy of the neural network classifier. 
With additional time, it would be worthwhile to explore additional neural network architectures or even different types of classifiers. Support vector machines, decision trees, or logistic regression could be investigated for the classification task. With regards to the textual semantic similarity between the job offers and the skills and activities, different pre-trained NLP encoders could be deployed, evaluated and compared. Finally, more research could be done to improve the cosine distance cutoff values to distinguish between the relevant and irrelevant skills and activities within the job offers text. With additional time, it is possible that choosing different cut off values based on the length of the job offer text, the type of sector the job is in, or the canton of the job offer could offer better results. All in all, this research has shown great promise and has demonstrated the abilities of machine learning and NLP to help researchers gain a better understanding of the current state of environmentally focused job opportunities within Switzerland.  
