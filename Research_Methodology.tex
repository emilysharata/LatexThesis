\chapter{Presentation of research problem}
\label{chap:procedure}

Like many other countries globally, Switzerland is interested in the development and evolution of the sustainable and environmentally-focused sector of the job market. In particular, the Canton of Geneva has made an effort to fund research to understand the present situation of this sector.  The aim of the research detailed in this thesis is to utilize NLP techniques to analyze a corpus of environmentally related job offers published online in Switzerland between the years 2017-2021. This project is a collaboration among Geneva’s Département du territoire (DT), the Institute for Environmental Sciences at the University of Geneva, and l'Haute \'Ecole Du Paysage, D'Ing\'enierie et D'Architecture De Gen\`eve (HEPIA). Together, these three universities and public authorities form the GE-EN-VIE network that is intended to support knowledge and communication on the environment and improve the quality of life in the Canton of Geneva. The actions taken by the network are designed to support the ``climate, energy, and biodiversity'' pillar of the Federal Office for Spatial Development’s 2030 Strategy for Sustainable Development. One of these concerns revolves around the dynamics and the evolution of the environmentally-focused labor market in Switzerland. The GE-EN-VIE network has found it necessary to gain a deeper understanding of the sector in order to better prepare job seekers for the market, to measure the presence of environmental job functions in Switzerland, and to assess the future needs to continue to support companies on their journeys to becoming more ecologically sustainable.

	Therefore, the aim  of this research is to analyze the specific skills and activities that are present in the corpus of job offers, building upon work that has been headed by Marie-José Genolet Viaccoz, a leader within the Institute for Environmental Sciences. Over 8000 job offers were obtained by the company X28, who provide detailed labor market data that were selected for their environmental character. A preliminary semantic analysis was performed by the GE-EN-VIE researchers to gain an understanding of the most common skills and activities in a subset of the job offers. These skills and activities were then compared and classified according to those organized by O*Net, a free online database that contains hundreds of occupational definitions. In this way, the researchers were able to link pre-defined skills and activities to the job offers, and measure their evolution over time. This research will build upon the work already completed by the GE-EN-VIE team with the objective of extracting the most frequent activities contained in the job offers, and matching them to the environmentally-related skills and activities contained in the O*Net-extracted data set, which provides a standard for each skill and activity.
